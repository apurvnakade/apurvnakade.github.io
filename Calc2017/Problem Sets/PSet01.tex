\documentclass[11pt, a4paper, oneside]{amsart}


\usepackage{amsmath}
\usepackage{amsthm}
\usepackage{amssymb}


\setlength{\topmargin}{-10mm}
\setlength{\textheight}{235mm}
\setlength{\oddsidemargin}{-3mm}
\setlength{\textwidth}{165mm}
\setlength{\footskip}{10mm}



\usepackage{fancyhdr}
\pagestyle{fancy}


\newcommand{\mytitle}{Homework 1 -- Selected Solutions }


\lhead{\scshape Apurva Nakade}
\rhead{\scshape Honors Single Variable Calculus}
\renewcommand*{\thepage}{\small\arabic{page}}
\title{Problem Set 01}

\begin{document}

\maketitle
\thispagestyle{fancy}

\section*{Day 1}
\vspace{1em}

\noindent {\textbf \S1.}
Explain in your own words how the provisional definition implies the $\epsilon-\delta$ definition of limit.\\


\textbf{Provisional definition:} The function $f$ approaches a limit $l$ near $a$, if we can make $f(x)$ as close to $a$ as we like to $l$ by requiring that $x$ be sufficiently close to, but unequal to, $a$.\\

\textbf{$\epsilon-\delta$ definition:} The function $f$ {approaches a limit} $l$ near $a$, if for every $\epsilon > 0$ there is some $\delta > 0$ such that, for all $x$, if $0 < |x - a| < \delta$, then $|f(x) - l| < \epsilon$.\\\\


\noindent {\textbf \S2.}
For each of the following functions $f$ and real numbers $a$,
\begin{itemize}
	\item Guess the limit $L = \lim \limits _ {x \rightarrow a} f(x)$.
	\item Find a $\delta$ corresponding to $\epsilon = 0.1$ in the $\epsilon - \delta$ definition of limit.
	\item For a real number $L' \neq L$ explain why you cannot find any $\delta$ that will work.
	\item Find a $\delta$ corresponding to an arbitrary real number $\epsilon$ and \textbf{prove} that $L$ is indeed the limit.\\
\end{itemize}

\begin{enumerate}
	\item $f(x)=x+2, \hspace{10pt} a=0$
	\item $f(x)=7x+ 2, \hspace{10pt} a=0$
	\item $f(x)=x^2-2, \hspace{10pt} a=0$
	\item $f(x)=\ln(x), \hspace{10pt} a=1$
\end{enumerate}









%%%%%%%%%%%%%%%%%%%%%%%%%%%%%%%%%%%%%%%%%%%%%%%%%%%%%%%%%%%%%%%
%%%%%%%%%%%%%%%%%%%%%%%%%%%%%%%%%%%%%%%%%%%%%%%%%%%%%%%%%%%%%%%
\newpage\section*{Day 2}
\vspace{1em}

\noindent {\textbf \S1.}
Explain in your own words how the provisional definition implies the $\epsilon-\delta$ definition of limit.\\


\textbf{Provisional definition:} The function $f$ approaches a limit $l$ near $a$, if we can make $f(x)$ as close to $a$ as we like to $l$ by requiring that $x$ be sufficiently close to, but unequal to, $a$.\\

\textbf{$\epsilon-\delta$ definition:} The function $f$ {approaches a limit} $l$ near $a$, if for every $\epsilon > 0$ there is some $\delta > 0$ such that, for all $x$, if $0 < |x - a| < \delta$, then $|f(x) - l| < \epsilon$.\\\\


\noindent {\textbf \S2.}
For each of the following functions $f$ and real numbers $a$,
\begin{itemize}
	\item Guess the limit $L = \lim \limits _ {x \rightarrow a} f(x)$.
	\item Find a $\delta$ corresponding to $\epsilon = 0.1$ in the $\epsilon - \delta$ definition of limit.
	\item For a real number $L' \neq L$ explain why you cannot find any $\delta$ that will work.
	\item Find a $\delta$ corresponding to an arbitrary real number $\epsilon$ and \textbf{prove} that $L$ is indeed the limit.\\
\end{itemize}

\begin{enumerate}
	\item $f(x)=x+2, \hspace{10pt} a=0$
	\item $f(x)=7x+ 2, \hspace{10pt} a=0$
	\item $f(x)=x^2-2, \hspace{10pt} a=0$
	\item $f(x)=\ln(x), \hspace{10pt} a=1$
\end{enumerate}









%%%%%%%%%%%%%%%%%%%%%%%%%%%%%%%%%%%%%%%%%%%%%%%%%%%%%%%%%%%%%%%
%%%%%%%%%%%%%%%%%%%%%%%%%%%%%%%%%%%%%%%%%%%%%%%%%%%%%%%%%%%%%%%
\newpage\section*{Day 3}
\vspace{1em}

\noindent {\textbf \S1.}
Explain in your own words how the provisional definition implies the $\epsilon-\delta$ definition of limit.\\


\textbf{Provisional definition:} The function $f$ approaches a limit $l$ near $a$, if we can make $f(x)$ as close to $a$ as we like to $l$ by requiring that $x$ be sufficiently close to, but unequal to, $a$.\\

\textbf{$\epsilon-\delta$ definition:} The function $f$ {approaches a limit} $l$ near $a$, if for every $\epsilon > 0$ there is some $\delta > 0$ such that, for all $x$, if $0 < |x - a| < \delta$, then $|f(x) - l| < \epsilon$.\\\\


\noindent {\textbf \S2.}
For each of the following functions $f$ and real numbers $a$,
\begin{itemize}
	\item Guess the limit $L = \lim \limits _ {x \rightarrow a} f(x)$.
	\item Find a $\delta$ corresponding to $\epsilon = 0.1$ in the $\epsilon - \delta$ definition of limit.
	\item For a real number $L' \neq L$ explain why you cannot find any $\delta$ that will work.
	\item Find a $\delta$ corresponding to an arbitrary real number $\epsilon$ and \textbf{prove} that $L$ is indeed the limit.\\
\end{itemize}

\begin{enumerate}
	\item $f(x)=x+2, \hspace{10pt} a=0$
	\item $f(x)=7x+ 2, \hspace{10pt} a=0$
	\item $f(x)=x^2-2, \hspace{10pt} a=0$
	\item $f(x)=\ln(x), \hspace{10pt} a=1$
\end{enumerate}




\end{document}

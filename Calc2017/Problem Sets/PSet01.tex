\documentclass[9pt, a4paper, oneside]{amsart}


\usepackage{enumitem}
\usepackage{parskip}
\usepackage{fancyhdr}
\pagestyle{fancy}

\newlist{questions}{enumerate}{1}
\setlist[questions, 1]{label = \bf Q.\arabic*., itemsep=1em}


% \setlength{\topmargin}{-10mm}
% \setlength{\textheight}{235mm}
% \setlength{\oddsidemargin}{-3mm}
% \setlength{\textwidth}{165mm}
% \setlength{\footskip}{10mm}



\newcommand{\mytitle}{Homework 1 -- Selected Solutions }


\lhead{\scshape Apurva Nakade}
\rhead{\scshape Honors Single Variable Calculus}
\renewcommand*{\thepage}{\small\arabic{page}}
\title{Problem Set 01}

\begin{document}

\maketitle
\thispagestyle{fancy}


\section*{Part 1}

The hardest step in writing an $ \epsilon-\delta$ proof is the first step, once you know how to start the proof the rest is mechanical. \\

\begin{questions}

	\item Explain in your own words how the provisional definition is equivalent to the more rigorous $\epsilon-\delta$ definition of limit.

	\begin{description}
		\item[Provisional definition] The function $f$ approaches a limit $L$ near $a$ i.e. $\lim \limits _ {x \rightarrow a} f(x) = L$, if we can make $f(x)$ as close to $L$ as we like by requiring that $x$ be sufficiently close to, but unequal to, $a$.
		\item[$\epsilon-\delta$ definition] The function $f$ {approaches a limit} $L$ near $a$ i.e. $\lim \limits _ {x \rightarrow a} f(x) = L$, if for every $\epsilon > 0$ there is some $\delta > 0$ such that, for all $x$, if $0 < |x - a| < \delta$, then $|f(x) - L| < \epsilon$.
	\end{description}

	% \item Another very useful way to rewrite the $\epsilon - \delta$ definition is the following: The function $f$ {approaches a limit} $l$ near $a$, if for every $\epsilon > 0$ there is some $\delta > 0$ such that, for all $x$, if $0 < |h| < \delta$, then $|f(a + h) - l| < \epsilon$. \\
	% Explain how this is equivalent to the standard $\epsilon-\delta$ definition.
	\item For each of the following functions $f$ and real numbers $a$,
	\begin{itemize}
		\item Guess the limit $L = \lim \limits _ {x \rightarrow a} f(x)$.
		\item Find a $\delta$ corresponding to $\epsilon = 0.1$ in the $\epsilon - \delta$ definition of limit.
		      % \item For a real number $L' \neq L$ explain why you cannot find any $\delta$ that will work.
		\item Find a $\delta$ corresponding to an arbitrary real number $\epsilon$ and \textbf{prove} that $L$ is indeed the limit.

		      \begin{enumerate}
		      	\item $f(x)=x+2, \hspace{10pt} a=0$
		      	\item $f(x)=7x+ 2, \hspace{10pt} a=0$
		      	\item $f(x)=x^2-2, \hspace{10pt} a=0$
		      	\item $f(x)=1/x, \hspace{10pt} a=1$
		      \end{enumerate}
	\end{itemize}
\end{questions}








%%%%%%%%%%%%%%%%%%%%%%%%%%%%%%%%%%%%%%%%%%%%%%%%%%%%%%%%%%%%%%%
%%%%%%%%%%%%%%%%%%%%%%%%%%%%%%%%%%%%%%%%%%%%%%%%%%%%%%%%%%%%%%%
\newpage\section*{Part 2}

It is tedious to use $ \epsilon-\delta $ proofs in practice, instead we use Theorem 2 and it's analogues.\\

\begin{questions}[resume]
	\item
	\begin{enumerate}
		\item Show that for every $ \epsilon_1, \epsilon_2 \in \mathbb{R}$ the following holds
		      \begin{align*}
		      	|\epsilon_1 - \epsilon _ 2| \le |\epsilon_1| + |\epsilon_2|
		      \end{align*}
		\item For which values of $ \epsilon_1, \epsilon_2 $ does equality hold?
		\item Using the $\epsilon-\delta$ definition of limit to prove that if $\lim \limits _ {x \rightarrow a} f(x) = l$ and $\lim \limits _ {x \rightarrow a} g(x) = m$ then $$\lim \limits _ {x \rightarrow a} \left(f(x)- g(x)\right) = l-m$$
	\end{enumerate}

	\item For the function 	$$ f(x) = \begin{cases} 0 & \mbox{ if } x < 0 \\ 1 & \mbox{ otherwise }\end{cases}$$
	determine, with proof, the limits $ \lim \limits_{x \rightarrow 0^+}$, $ \lim \limits_{x \rightarrow 0^-}$. How will you answers change if we replace $ x<0$ by $ x\le 0$ in the definition of $ f(x)$?


	\item
	\begin{enumerate}
		\item Determine, with proof, $\lim \limits_{x \rightarrow \infty} 1/x$.
		\item Prove that for no real number $l$ do we have $\lim \limits_{x \rightarrow 0^+} 1/x = l$.
	\end{enumerate}


	\item Give examples to show that the following definitions of $\lim \limits _ {x \rightarrow a} f(x) = L$ are not correct i.e. find functions $ f$ which are continuous but do not satisfy the following conditions. (Hint: Think graphically)
	\begin{itemize}
		\item For all $ \delta > 0$ there exists an $ \epsilon > 0$ such that if $ 0<|x-a|< \delta$ then $ |f(x) - L|< \epsilon$.
		\item For every $\epsilon > 0$ there is some $\delta > 0$ such that, for all $x$, if $|f(x) - L| < \epsilon$ then $0 < |x - a| < \delta$.
	\end{itemize}

\end{questions}
Trigonometric functions, exponential functions and logarithms are continuous wherever they are defined. We will assume this fact without proof for now and come back to it later.











%%%%%%%%%%%%%%%%%%%%%%%%%%%%%%%%%%%%%%%%%%%%%%%%%%%%%%%%%%%%%%%
%%%%%%%%%%%%%%%%%%%%%%%%%%%%%%%%%%%%%%%%%%%%%%%%%%%%%%%%%%%%%%%
\newpage\section*{Part 3}

To understand continuity it is equally important to understand discontinuity.
\begin{questions}[resume]
	\item
	\begin{enumerate}
		\item Let $ n$ be a positive integer. Use the $ \epsilon-\delta$ definition to prove that the function $f(x)=x^n$ is continuous at 0.

		\item Use Theorem 2 to prove that every polynomial $p(x)$ is continuous at 0.

		\item Prove that a function $ f(x)$ is continuous at a real number $a$ if and only if the function $ g(x) = f(x + a)$ is continuous at $0$.

		\item Prove that $x^n$ is continuous at every real number $ a $.

		\item Use Theorem 2 to prove that every polynomial $p(x)$ is continuous at every real number $ a $.

		\item What are the real numbers at which the ratio of two polynomials $\dfrac{p(x)}{q(x)}$ is continuous?
	\end{enumerate}


	\item Prove that the function
	$$ f(x) = \begin{cases} 0 & \mbox{ if $ x$ is rational } \\ 1 & \mbox{ otherwise }\end{cases}$$
	is discontinuous everywhere. What can you say about the continuity of the function $ g(x) = x .f(x)$?
\end{questions}








\end{document}
